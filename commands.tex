% pebl-structure http://yuml.me/edit/44a2d769
% pebl-test-high-level 	http://yuml.me/edit/d3df3127


% Disable single lines at the start of a paragraph (Schusterjungen)
\clubpenalty = 10000
%
% Disable single lines at the end of a paragraph (Hurenkinder)
\widowpenalty = 10000 \displaywidowpenalty = 10000


\crefname{figure}{Figure}{Figures}
\Crefname{figure}{Figure}{Figures}

\newcommand{\hypothesisChapter}[2]{
\begin{mdframed}[backgroundcolor=black!10,rightline=false,leftline=false,bottomline=false,topline=false]
In this chapter, {hypothesis~#1} \textit{(``#2'')} is supported.
\end{mdframed}
}

\newcommand{\hypothesisSection}[2]{
\begin{mdframed}[backgroundcolor=black!10,rightline=false,leftline=false,bottomline=false,topline=false]
In this section, {hypothesis~#1} \textit{(``#2'')} is supported.
\end{mdframed}
}

\newcommand{\hypothesisDefinition}[2]{
\begin{mdframed}[backgroundcolor=black!10]
\textbf{\emph{Hypothesis #1}} \textit{#2}
\end{mdframed}
}


% ensures this is always correctly aligned
\let\brokenmarginnote\marginnote
\renewcommand{\marginnote}[1]{\leavevmode\brokenmarginnote{{\scriptsize \foreignlanguage{british}{\begin{spacing}{1.025}%
\vspace{-\baselineskip}
#1%
\end{spacing}}}}\ignorespaces}

\newcommand{\userstory}[3]{\textbf{As a} #1,\newline\textbf{I want to} #2 \newline \textbf{so that I can} #3.}

\definecolor{hellgrau}{gray}{0.9}

\newcommand{\myrowcolour}{\rowcolor[gray]{0.925}}
\usepackage{theorem}
\theoremstyle{break}

\newtheorem{defi}{Definition}[chapter]

\newcommand{\definitioncited}[3]{
\begin{mdframed}[linecolor=black!50,linewidth=5pt,bottomline=false,topline=false,innerrightmargin=.5cm]
\begin{defi}[#1]``#2''~\emph{#3}\end{defi}
\end{mdframed}
}

\newcommand{\definitionown}[2]{
\begin{mdframed}[linecolor=black!50,linewidth=5pt,bottomline=false,topline=false,innerrightmargin=.5cm]
\begin{defi}[#1] #2\end{defi}
\end{mdframed}
}

\newcommand{\lastaccessed}[0]{, visited DATE}


% tip by http://tex.stackexchange.com/a/73710/9075
\makeatletter
 \newcommand{\labelname}[1]{% \labelname{<stuff>}
  \def\@currentlabelname{#1}}%
\makeatother

% always use the compact variant instead
\renewenvironment{description}[0]{\begin{compactdesc}}{\end{compactdesc}}


% patterns
\newcommand{\patternOneDefinition}{%
Pattern One~(P1)%
\labelname{(P1)}\label{P1}%
}
\newcommand{\patternOneReference}{Pattern One~\nameref{P1}\xspace}


% basedupon
%
% cited papers
\newcommand{\based}[1]{
\begin{flushright}
\textit{#1}
\end{flushright}
}
\newcommand{\baseduponchapter}[1]{
\based{Parts of this chapter have been taken from #1.}
}
\newcommand{\baseduponsection}[1]{
\based{Parts of this section have been taken from #1.}
}


% pattern
%
% name, problem, solution, example, relations
\newcommand{\pattern}[5]{
\begin{mdframed}[backgroundcolor=black!15,rightline=false,leftline=false,topline=false,bottomline=false,splittopskip=0.5cm,nobreak=true]
{\textbf{Pattern #1}}\vspace{10pt}
{\footnotesize
  \begin{description}
  	\item[Problem] #2
  	\item[Solution] #3
  	\item[Example] #4
  	\ifthenelse{\equal{#5}{}}{}{\item[Relations] #5} 	
  \end{description}
  }
\end{mdframed}
}

\newcommand{\myquote}[2]{%
\epigraphhead[30]{\epigraph{#1}{\emph{#2}}}%
}%










\lstdefinestyle{xmlStyle}{%
  language=XML,%
  basicstyle=\ttfamily\scriptsize,%
  commentstyle=\itshape,%
  backgroundcolor=\color{hellgrau},%
  keywordstyle=\bfseries,%
  showstringspaces=false,%
  numbers=left,%
  numberstyle=\tiny,%
  stepnumber=1,%
  numbersep=5pt,%
  extendedchars=true,%
  xleftmargin=0.5em,%
  xrightmargin=0em,%0.5em
  lineskip=-1pt,%
 	tabsize=1,%
  breaklines,%
  morekeywords={encoding,
  	    xs:schema,xs:element,xs:complexType,xs:sequence,xs:attribute},
  	    morestring=[b]",
  	    morecomment=[s]{<?}{?>},
  	    morecomment=[s][\color{orange}]{<!--}{-->},
  	    keywordstyle=\color{blue},
  	    stringstyle=\color{red},
  	    tagstyle=\color{blue},
}

%
% neues environment fuer XML-Sourcecode
% #1 = ueberschrift
% #2 = Label
%
\lstnewenvironment{xmlCode}[2]%
{\lstset{style=xmlStyle,caption={#1},label=#2}}{}


\lstdefinestyle{pseudoStyle}{%
  basicstyle=\ttfamily\scriptsize,%
  commentstyle=\itshape,%
  backgroundcolor=\color{hellgrau},%
  keywordstyle=\bfseries,%
  showstringspaces=false,%
  numbers=left,%
  numberstyle=\tiny,%
  stepnumber=1,%
  numbersep=5pt,%
  extendedchars=true,%
  xleftmargin=0.5em,%
  xrightmargin=0em,%0.5em
  lineskip=-1pt,%
  tabsize=1,%
  breaklines,%
  morecomment=[l][\color{orange}]{//},
  morekeywords={for, each, in, into, from, is, if, return, def, let, task, call, else, then, remote, to, has, wait, until, via},
  keywordstyle=\color{blue}
}

%To get rid of overfull hobx warnings that really don't matter
\hfuzz=2pt

%
% neues environment fuer pseudo-Sourcecode
% #1 = ueberschrift
% #2 = Label
%
\lstnewenvironment{pseudoCode}[2]%
{\lstset{style=pseudoStyle,caption={#1},label=#2}}{}


% for nicer back references
\renewcommand*{\backref}[1]{}
\renewcommand*{\backrefalt}[4]{%
    \ifcase #1 (Not cited.)%
    \or        (Cited on page~#2.)%
    \else      (Cited on pages~#2.)%
    \fi}

\makeatletter
\def\ll@defi{%
  \protect\numberline{\csname the\thmt@envname\endcsname}%
  \ifx\@empty\thmt@shortoptarg
    \thmt@thmname
  \else
    \thmt@shortoptarg
  \fi}
\def\l@thmt@defi{} 
\makeatother
